%---------------------------------------------------------------------
% Imagens pré-textuais (precisam estar no mesmo diretório deste arquivo .tex)
%---------------------------------------------------------------------

\logo{formato/logo_uerj_cinza.png}
\marcadagua{formato/marcadagua_uerj_cinza.png}{1}{160}{255}

%---------------------------------------------------------------------
% Informações da instituição
%---------------------------------------------------------------------

\instituicao{Universidade do Estado do Rio de Janeiro}
            {Centro de Tecnologia e Ciências} 
            {Faculdade de Engenharia} 
            {Departamento de Engenharia Cartográfica} 
             

%---------------------------------------------------------------------
% Informações da autoria do documento
%---------------------------------------------------------------------
%O48
\autor{Luiz Otavio Soares de }
      {Oliveira}
      {LO}

\titulo{Medição automática de pontos de costura para conjuntos de imagens e sua aplicação ao Projeto E-Foto}
% se não for usar a quarta palavra chave, deixar o campo vazio: {}
\palavraschaves{Fotogrametria}
               {Costura de imagens}
               {Projeto E-Foto}
               {OpenCV}

\title{Automatic measurement of stitch points for image sets and their application to the E-Foto Project.}
\keywords{Photogrammetry}
         {Image stitch}
         {E-Foto Project}
         {OpenCV}

\orientador{Prof. Me.} 
           {Irving}{da Silva Badolato} 
           {Faculdade de Engenharia -- UERJ} 

%coorientador é opcional
%\coorientador{Prof. Dr.} 
%             {Luiz Henrique}{Guimarães Castiglione} 
%             {Faculdade de Engenharia -- UERJ} 

%---------------------------------------------------------------------
% Grau pretendido (Doutor, Mestre, Graduado, Bacharel, Licenciado, Engenheiro), % Curso (Engenharia Cartográfica, Engenharia Ambiental, Engenharia Elétrica) e 
% Gênero (Masculino, Feminino)
%---------------------------------------------------------------------

\grau{Engenheiro}  
\curso{Engenharia Cartográfica}
\genero{Masculino}

% área de concentração é opcional
%\areadeconcentracao{área}

%---------------------------------------------------------------------
% Informações adicionais (local, data e paginas)
%---------------------------------------------------------------------

\local{Rio de Janeiro} 
\data{13}{maio}{2022} % <= Atualize com a data da defesa