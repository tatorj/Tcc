\chapter*{Conclusão}
%===================================================================== 

Após diversos testes realizados com os métodos estudados para a solução a que este trabalho se aplica, se tratando de recursos de código aberto e passíveis de utilização no Projeto E-Foto, foi construído um programa de interface de linha de comando que pode ser integrado em versões futuras. O método SURF, apesar de oferecer boas respostas, não deve ser aproveitado com facilidade por conta de sua patente ainda vigente. Entende-se que deva permanecer válida até meados de 2034. Por tal motivo o código fonte resultante está preparado para desconsiderar esta opção quando compilado em distribuições mais comuns da OpenCV.

Dentre os demais métodos que foram estudados, o AKaze e o SIFT demonstraram maior tempo de execução quando comparados com o método ORB. Na comparação entre esses métodos quanto ao tamanho alocado, o ORB se mostrou o método mais eficiente, apesar de não haver muita discrepância entre a quantidade de memória requerida para os resultados entre este e o AKaze. Em comparação com o tamanho utilizado pelo método SIFT, este necessitou de maior alocação de memória para a descrição das feições, tendo um acréscimo de aproximadamente 300\% no tamanho alocado por ponto.

A correlação e a verificação geométrica de pares depende principalmente da quantidade de dados passados a elas.  Conforme foi discorrido no capítulo \ref{results}, a escolha do método de correlação afeta em grande parte a velocidade de processamento da solução, portanto, indicamos o uso da correlação por FLANN já que este apresenta melhor desempenho quando realiza a análise de grande quantidades de feições, o que é bastante comum no tipo de análise que este trabalho está realizando. Vale enfatizar que o uso de correlação cruzada por força bruta pode gerar soluções com menor erro geométrico, mas requer controle para limitar o número de pontos chaves detectados, haja visto o seu desempenho, e maios estudos dos impactos que podem estar relacionados às restrições aplicáveis.

A verificação geométrica por sua vez não apresentou grandes discrepâncias nos métodos estudados quando analisados os tempos e tamanhos alocados. Ainda assim recomenda-se a execução de mais estudos que abordem o número de respostas e a própria solução geométrica para classificar os pares analisados. Na fotogrametria é comum, por exemplo, que haja uma expectativa de sobreposição das fotos de uma mesma faixa de voo e uma sobreposição entre faixas e o atendimento destas expectativas não são atualmente verificadas.

Todos os critérios propostos puderam ser estudados com os resultados numéricos e visuais distintos que foram apresentados. Contudo, seria inviável endereçar todo o conjunto de testes que foi realizado no capítulo de resultados. Ressalta-se que foram feitos testes unitários para cada variação esperada dos argumentos de entrada para a solução. Dados externos ao projeto, além do conjunto de dados apresentado na seção \ref{conjdados}, fizeram-se necessários para a análise de escalabilidade e testes de hipóteses levantadas durante o projeto. Isto não invalida trabalhos futuros com mais detalhes da manutenção da escalabilidade para grandes blocos de imagens. 

Conclui-se, portanto, que dentre os métodos expostos dois se destacam nos quesitos estudados: o ORB, por sua velocidade de processamento e distribuição espacial dos pontos de costura; e o SIFT, que apesar de ser um método mais custoso computacionalmente apresenta o melhor resultado quando comparamos o volume de pontos de costura e menor erro. Sugere-se ainda que os resultados aqui apresentados sirvam de incentivo para trabalhos futuros como: uma extensão para visualização blocos de imagens fotogramétricas, em 2D como num foto-índice ou em 3D para a exibição simultânea de fotos e pontos triangulados pela solução fotogramétrica; a criação ou adaptação de uma ou mais interfaces gráficas que incorporem o código derivado deste trabalho aos módulos do e-foto; e a realização de estudos para uso de OpenCV na extração automatizada de linhas e polígonos que sirvam como linhas de quebra no e-foto.