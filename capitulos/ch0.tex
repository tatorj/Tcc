\chapter*{Introdução}

% Contexto geral:
A história da cartografia e do mapeamento se entrelaça com a da humanidade, visto que este meio de informação é essencial para nos situarmos geograficamente. No entanto, uma visão real, a nadir, para geração de mapas melhores foi por muito tempo um feito quase impossível. 
Mesmo com o advento dos voos controlados usando balões de ar quente, ainda era complexo registrar a informação a ser trabalhada. Então com a invenção e popularização da câmera fotográfica, na primeira metade do século XIX, o inventor francês Aimé Laussedat, percebeu as aplicações que essa nova tecnologia poderia ter no campo da cartografia. Uma das maiores inovações viria no campo do mapeamento, assim criando uma nova área na cartografia que viria a ser conhecida como fotogrametria.

% Objeto do trabalho:
Com o início da utilização de fotografias para mapeamento, surgiu a questão sobre como conectar várias imagens, umas às outras, para tornar possível a aquisição de mais informações sobre o objeto fotografado. Dentre as práticas adotadas tornou-se comum a escolha de pontos homólogos, que são partes dos mesmos objetos na cena fotografada visíveis em fotos diferentes, para prender estas imagens umas às outras usando-os como pontos de costura. Por sua vez, costurar conjuntos de imagens manualmente é um processo extremamente longo que aumenta em complexidade rapidamente conforme cresce o número de imagens disponíveis.

Tal costura era inicialmente realizada fisicamente ao se revelar as imagens e criar o mosaico literalmente costurando fisicamente as fotos para que o modelo se tornasse íntegro. Com a necessidade de medidas mais rigorosas e a evolução tecnológica foram criadas máquinas de restituição óptico-mecânicas, nas quais as imagens, que anteriormente eram costuradas, passaram a serem visualizadas por meio de diafilmes e com elas o restituidor passou a poder ajusta-las para obter visão estereoscópica e, assim, realizar as medidas. Com suportes mecânicos era possível inclusive plotar pontos e vetorizar outras geometrias sem a necessidade de costurar fisicamente as imagens para realização desses procedimentos.

Recentemente, com a evolução da computação e das câmeras, passou-se a utilizar imagens digitais em ambientes computadorizados. Por sua vez, a costura de imagens digitais já é um processo bem documentado e automatizado na área de visão computacional. Esse campo compartilha métodos computacionais de aprendizado de máquina, do campo de inteligência artificial, para realizar entre outras coisas a análise de entradas visuais com interpretação de imagens e a reconstrução da geometria das cenas.

% Motivadores: 
Para fins acadêmicos, no contexto de desenvolvimento de ferramentas para ensino de fotogrametria, o Projeto E-foto possui entre seus produtos um software, denominado e-foto, que visa ser uma plataforma de fotogrametria digital em software livre, ele é desenvolvido no Laboratório de Fotogrametria e Sensoriamento Remoto da Faculdade de Engenharia da Universidade do Estado do Rio de Janeiro, desde 2004, tendo sido implementado por professores e alunos desde seu início. Em sua trajetória observa-se que esse tipo de implementação gera lacunas, onde certas automações não são incluídas, por não serem imediatamente necessárias para o funcionamento do projeto ou para que os passos intermediários possam ser estudados e praticados formando pensamento crítico sobre os resultados obtidos para análise de resultados finais. Logo, trabalhos repetitivos como o descrito acima ainda são feitos manualmente, não existindo alternativa viável para acelerar a produção depois de compreender o assunto, e, como consequência, aumentando a carga horária da realização de trabalhos.

% Objetivos:
Este trabalho se propõe a realizar um comparativo entre diversas técnicas comuns no processo de obtenção automática de pontos de costura, tais como detecção e descrição de pontos chave, correlação e verificação geométrica de pontos correlatos, utilizando-se das atuais tecnologias no campo de visão computacional. Estima-se que tal ensaio deve servir de balizador para determinar quais destas técnicas melhor se adéquam ao Projeto E-foto do ponto de vista da qualidade, desempenho e escalabilidade usando imagens aéreas. Entre os resultados materializáveis está uma proposta de solução para o problema que possa ser integrada ao software em versões futuras.

% Estrutura do trabalho:
Além desta introdução e da conclusão, o desenvolvimento deste trabalho apresenta divisão em 3 capítulos, a saber: A fundamentação teórica, que apresenta as áreas de fotogrametria e visão computacional; a metodologia, que apresenta os passos necessários à confecção da solução para o problema proposto; e os resultados, onde diversos dados comparativos são analisados.
