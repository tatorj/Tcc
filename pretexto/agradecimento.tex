\pretextualchapter{Agradecimentos}

Agradeço primeiramente a UERJ que apesar do tempo nunca duvidou do meu potencial, possibilitou minha formação e me ensinou a ser uma pessoa melhor do que quando ingressei. Ao meu incansável orientador que mesmo ocupado não me abandonou a deriva neste mar de direções a serem seguidas e a sua família por entender e aceitar nossos horários loucos e longos de reunião. Aos professores do departamento de Cartografia por seus ensinamentos.

A minha família que me apoiou nessa trajetória de pontos, costuras e ligações até formar uma imagem única e clara de quem sou e para onde vou. Em especial minha mãe Maria da Conceição, a pessoa mais forte que eu conheço que nunca se deixou levar pelas adversidades e me ensinou a ser a pessoa que sou hoje, minha irmã Maria Caroliny, que sempre me apoiou e me desafiou a ser melhor que eu fui ontem, meu pai Luiz Antônio, que me apoiou mesmo que as vezes eu não entendesse seu apoio, e meu namorado Renan, cujo apoio, amor incondicional e ombro amigo me salvaram diversas vezes nesses anos juntos.

Aos amigos que a vida trouxe e levou de acordo com seus caprichos, mas principalmente aos que se mantém que são amizades que pretendo cultivar para toda a vida. Especialmente, meus amigos mais antigos Thomaz, Ronald e Márcio. A Daniele que me apoiou em diversas descobertas da vida, e Renan Monteiro que apesar de sua jovem estadia em minha vida já é presença garantida no resto dela.

Aos profissionais da clínica Sinte em especial Jaqueline e Lilian, que cuidaram da minha coluna com seus diversos maquinários de tortura, independente de minhas reclamações, e assim possibilitaram as longas horas de trabalho que este trabalho demandou.

E finalmente agradeço a todos os meus cães, que me fizeram companhia nas longas horas dessa jornada.

A todos o meu mais profundo agradecimento.