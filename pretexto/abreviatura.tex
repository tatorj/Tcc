\pretextualchapter{Lista de abreviaturas e siglas}
\abreviatura{API:        }   {Application Programming Interface}
\abreviatura{3D:        }   {Espaço tridimensional}
\abreviatura{2D:        }   {Espaço bidimensional}
\abreviatura{e-foto:        }   {Estação fotogramétrica digital educacional}
\abreviatura{UERJ:        }   {Universidade do Estado do Rio de Janeiro}
\abreviatura{LFSR:        }   {Laboratório de Fotogrametria e Sensoriamento Remoto}
\abreviatura{DSM:        }   {Digital Surface Model}
\abreviatura{DEM:        }   {Digital Elevation Model}
\abreviatura{GIS:        }   {Geographic Information Sistem}
\abreviatura{IBGE:        }   {Instituto Brasileiro de Geografia e Estatística}
\abreviatura{IBM:        }   {International Business Machines}
\abreviatura{IA:     }   {Inteligencia Artificial}
\abreviatura{BFM:    }   {Brute-Force Matcher}
\abreviatura{SURF:   }   {Speeded Up Robust Features}
\abreviatura{LSM:    }   {Least Squares Method}
\abreviatura{RANSAC: }   {Random Sample Consensus}
\abreviatura{LMEDS:  }   {Least-Median of Squares}
\abreviatura{RHO:    }   {Progressive Sample Consensus}
\abreviatura{OpenCV: }   {Open Computer Vision}
\abreviatura{SIFT:   }   {Scale Invariant Feature Transform}
\abreviatura{AKAZE:  }   {Accelerated Kaze }
\abreviatura{FED:    }   {Fast Explicit Diffusion}
\abreviatura{ORB:    }   {Oriented FAST and Rotated BRIEF}
\abreviatura{FAST:   }   {Features from Accelerated Segment Test}
\abreviatura{BRIEF:  }   {Binary Robust Independent Elementary Features}
\abreviatura{ODM:    }   {Open Drone Maps}
\abreviatura{OSGeo: }   {Open Source Geospatial Foundation}
\abreviatura{CUDA   }   {Computer Unified Device
Architecture}
\abreviatura{GPU:        }   {Graphics Processing Unit}
\abreviatura{FLANN:        }   {Fast Library for Approximate Nearest Neighbors}
\abreviatura{UML:        }   {Unified Modeling language}
\abreviatura{KD-Tree:        }   {Árvores de K-dimensões}
\abreviatura{STL:        }   {Standard Template Library}   
\abreviatura{ISPRS:  }   {International Society for Photogrammetry and Remoto Sensing}   


