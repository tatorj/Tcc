% ----------------------------------------------------------
% Apêndices (opcionais)
% ----------------------------------------------------------
% ---
% Inicia os apêndices
% ---
\appendix

%=====================================================================
\postextualchapter{ Código fonte }\label{codigofonte}

\section{Arquivo de configuração do projeto}
\subsection{src/CMakeLists.txt}
\lstinputlisting[language={}]{src/CMakeLists.txt}

\section{Arquivos de cabeçalho}
\subsection{src/macros.hpp}
\lstinputlisting[]{src/macros.hpp}
\subsection{src/point.hpp}
\lstinputlisting[]{src/point.hpp}
\subsection{src/image.hpp}
\lstinputlisting[]{src/image.hpp}
\subsection{src/pair.hpp}
\lstinputlisting[]{src/pair.hpp}
\subsection{src/control.hpp}
\lstinputlisting[]{src/control.hpp}

\section{Arquivos de implementação}
\subsection{src/point.cpp}
\lstinputlisting[]{src/point.cpp}
\subsection{src/image.cpp}
\lstinputlisting[]{src/image.cpp}
\subsection{src/pair.cpp}
\lstinputlisting[]{src/pair.cpp}
\subsection{src/control.cpp}
\lstinputlisting[]{src/control.cpp}
\subsection{src/main.cpp}
\lstinputlisting[]{src/main.cpp}


%=====================================================================
\postextualchapter{ Padrões para definir regiões de interesse de gruber}\label{pattern}

\section{Implementação dos padrões em formato ASCII Portable GrayMap}
\subsection{Padrão 2-2-2 implementado no arquivo data/pattern0.pgm}
\lstinputlisting[basicstyle=\scriptsize\ttfamily,language={}]{data/pattern0.pgm}
\subsection{Padrão 3-3-3 implementado no arquivo data/pattern1.pgm}
\lstinputlisting[basicstyle=\scriptsize\ttfamily,language={}]{data/pattern1.pgm}
\subsection{Padrão 5-5-5 implementado no arquivo data/pattern2.pgm}
\lstinputlisting[basicstyle=\scriptsize\ttfamily,language={}]{data/pattern2.pgm}
\subsection{Padrão 3-2-3-2-3 implementado no arquivo data/pattern3.pgm}
\lstinputlisting[basicstyle=\scriptsize\ttfamily,language={}]{data/pattern3.pgm}

\begin{figure}[!ht]{12.7cm}
\subsection{Visualização dos padrões implementados}
  \caption{Padrões interpolados para imagens de 360x360 pixels}
  \subfloat[][]{\label{pattern0}
    \setlength{\fboxsep}{0pt}
    \fbox{\includegraphics[width=0.5\hsize]{data/pattern0.png}}}
  \subfloat[][]{\label{pattern1}
    \setlength{\fboxsep}{0pt}
    \fbox{\includegraphics[width=0.5\hsize]{data/pattern1.png}}}\\
  \subfloat[][]{\label{pattern2}
    \setlength{\fboxsep}{0pt}
    \fbox{\includegraphics[width=0.5\hsize]{data/pattern2.png}}}
  \subfloat[][]{\label{pattern3}
    \setlength{\fboxsep}{0pt}
    \fbox{\includegraphics[width=0.5\hsize]{data/pattern3.png}}}\\
  \legend{Visualização dos padrões de gruber implementados a saber:\\ \ref{pattern0} 2-2-2; \ref{pattern1} 3-3-3; \ref{pattern2} 5-5-5; \ref{pattern3} 3-2-3-2-3.}
  \source{O autor.}
\end{figure}